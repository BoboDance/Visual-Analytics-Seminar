%%Donut chart

\pgfkeys{%
/piechartthreed/.cd,
scale/.code                =  {\def\piechartthreedscale{#1}},
mix color/.code            =  {\def\piechartthreedmixcolor{#1}},
background color/.code     =  {\def\piechartthreedbackcolor{#1}},
name/.code                 =  {\def\piechartthreedname{#1}}}

\newcommand\piechartthreed[2][]{% 
   \pgfkeys{/piechartthreed/.cd,
     scale               = 1,
     mix color        = gray,
     background color = white,
     name             = pc} 
  \pgfqkeys{/piechartthreed}{#1}
  \begin{scope}[scale=\piechartthreedscale] 
  \begin{scope}[xscale=5,yscale=3] 
%     \path[preaction={fill=black,opacity=.8,
%         path fading=circle with fuzzy edge 20 percent,
%         transform canvas={yshift=-15mm*\piechartthreedscale}}] (0,0) circle (1cm);
     \path[preaction={fill=black,opacity=.8,
         path fading=circle with fuzzy edge 20 percent,
         transform canvas={yshift=-15mm*\piechartthreedscale}}] (0,0) circle (1cm);
    \fill[gray](0,0) circle (0.5cm);  
     \path[preaction={fill=\piechartthreedbackcolor,opacity=.8,
          path fading=circle with fuzzy edge 20 percent,
          transform canvas={yshift=-10mm*\piechartthreedscale}}] (0,0) circle (0.5cm);
     \pgfmathsetmacro\totan{0} 
     \global\let\totan\totan 
     \pgfmathsetmacro\bottoman{180} \global\let\bottoman\bottoman 
     \pgfmathsetmacro\toptoman{0}   \global\let\toptoman\toptoman 
     \begin{scope}[draw=black,thin]
     \foreach \an/\col [count=\xi] in {#2}{%
     \def\space{ } 
        \coordinate (\piechartthreedname\space\xi) at (\totan+\an/2:0.75cm); 
        \ifdim 180pt>\totan pt 
         \ifdim 0pt=\toptoman pt
            \shadedraw[left color=\col!20!\piechartthreedmixcolor,
                       right color=\col!5!\piechartthreedmixcolor,
                       draw=black,very thin] (0:.5cm) -- ++(0,-3mm) arc (0:\totan+\an:.5cm) 
                                                       -- ++(0,3mm)  arc (\totan+\an:0:.5cm);
            \pgfmathsetmacro\toptoman{180} 
            \global\let\toptoman\toptoman         
            \else
            \shadedraw[left color=\col!20!\piechartthreedmixcolor,
                       right color=\col!5!\piechartthreedmixcolor,
                       draw=black,very thin](\totan:.5cm)-- ++(0,-3mm) arc(\totan:\totan+\an:.5cm)
                                                        -- ++(0,3mm)  arc(\totan+\an:\totan:.5cm); 
          \fi
        \fi   
        \fill[\col!20!gray,draw=black] (\totan:0.5cm)--(\totan:1cm)  arc(\totan:\totan+\an:1cm)
                                     --(\totan+\an:0.5cm) arc(\totan+\an:\totan :0.5cm);     
       \pgfmathsetmacro\finan{\totan+\an}
       \ifdim 180pt<\finan pt 
         \ifdim 180pt=\bottoman pt
            \shadedraw[left color=\col!20!\piechartthreedmixcolor,
                       right color=\col!5!\piechartthreedmixcolor,
                       draw=black,very thin] (180:1cm) -- ++(0,-3mm) arc (180:\totan+\an:1cm) 
                                                       -- ++(0,3mm)  arc (\totan+\an:180:1cm);
            \pgfmathsetmacro\bottoman{0}
            \global\let\bottoman\bottoman
            \else
            \shadedraw[left color=\col!20!\piechartthreedmixcolor,
                       right color=\col!5!\piechartthreedmixcolor,
                       draw=black,very thin](\totan:1cm)-- ++(0,-3mm) arc(\totan:\totan+\an:1cm)
                                                        -- ++(0,3mm)  arc(\totan+\an:\totan:1cm); 
          \fi
        \fi
        \pgfmathsetmacro\totan{\totan+\an}  \global\let\totan\totan 
       } 
    \end{scope}
    \draw[thin,black](0,0) circle (0.5cm);
   \end{scope}  
\end{scope}
}

\newcommand*{\mytextstyle}{\sffamily\bfseries\color{black!85}}
\newcommand{\arcarrow}[3]{%
   % inner radius, middle radius, outer radius, start angle,
   % end angle, tip protusion angle, options, text
   \pgfmathsetmacro{\rin}{1.7}
   \pgfmathsetmacro{\rmid}{2.2}
   \pgfmathsetmacro{\rout}{2.7}
   \pgfmathsetmacro{\astart}{#1}
   \pgfmathsetmacro{\aend}{#2}
   \pgfmathsetmacro{\atip}{5}
   \fill[mygray, very thick] (\astart+\atip:\rin)
                         arc (\astart+\atip:\aend:\rin)
      -- (\aend-\atip:\rmid)
      -- (\aend:\rout)   arc (\aend:\astart+\atip:\rout)
      -- (\astart:\rmid) -- cycle;
   \path[
      decoration = {
         text along path,
         text = {|\mytextstyle|#3},
         text align = {align = center},
         raise = -1.0ex
      },
      decorate
   ](\astart+\atip:\rmid) arc (\astart+\atip:\aend+\atip:\rmid);
}


\begin{tikzpicture}
   \piechartthreed[scale=0.8,
                   background color=white,
                   mix color= darkgray]
                   {164.52/lightgray,98.64/red,96.84/red}
                   
\foreach \i in {1,...,3} { \draw[fill=black] (pc \i) circle (.5mm);}

\draw[black, align=right] (pc 1)  -- ++(6,0) coordinate (s1) node[anchor=south east] {\small Nicht Betroffen}
node[anchor=north east] {45.7\%};

\draw[black] (pc 2)   -- ++(-4,0) coordinate (s3)
node[anchor=south west] {\small Verdacht}
node[anchor=north west] {27.4\%}; 

\draw[black] (pc 3)   -- ++(4.5,0) coordinate (s2)
node[anchor=south east] {\small Betroffen}
node[anchor=north east] {26.9\%};  

\end{tikzpicture}
